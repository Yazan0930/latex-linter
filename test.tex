begin entries are indicated with, a black dot, a so-called bullet....

\documentclass{article}

%\begin{i temize}
\item List entries start with the \verb|\item| command.
\item Individual entries are indicated with a black dot, a so-called bullet.
\item The text in the entries may be of any length.
\begin{itemize}
\item List entries start with the \verb|\item| command.
\item Individual entries are indicated with a black dot, a so-called bullet.
\item The text in the entries may be of any length.
\begin{itemize}
\item List entries start with the \verb|\item| command.
\item Individual entries are indicated with a black dot, a so-called bullet.
\item The text in the entries may be of any length.
\end{itemize}
\end{itemize}
\end{itemize}

%this to this
\maketitle

\chapter{First Chapter}

\section{Introduction}

This is the first section.

Lorem  ipsum  dolor  sit  amet,  consectetuer  adipiscing
elit.   Etiam  lobortisfacilisis sem.  Nullam nec mi et
neque pharetra sollicitudin.  Praesent imperdietmi nec ante.
Donec ullamcorper, felis non sodales...

\section{Second Section}
Lorem ipsum dolor sit amet, consectetuer adipiscing elit.
Etiam lobortis facilisissem.  Nullam nec mi et neque pharetra
sollicitudin.  Praesent imperdiet mi necante...
\subsection{First Subsection}
Praesent imperdietmi nec ante. Donec ullamcorper, felis non sodales...
\section*{Unnumbered Section}
Lorem ipsum dolor sit amet, consectetuer adipiscing elit.
Etiam lobortis facilisissem
 $f(x) = f(x-1) + f(x-2)$
\end{document}